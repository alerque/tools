\setuppapersize [A5] [A5]

\definecolor [gutenred] [x=bf221f] % rubrication from digitized_Göttingen Gutenberg bible

\setupinteraction [
  state=start,
  color=gutenred, % rubricate, don’t viridificate
  contrastcolor=gutenred,
]

\setupcombinedlist[content][interaction=text,focus=standard]

\setupindenting[yes,next,medium]

%\showgrid
\setuphead[chapter][
  align=middle,
  number=no,
  style={\rm\tfa\setcharacterkerning[capitals]\WORD},
  before={\blank[5*line]},
  after={\blank[2*line,force]}
]

\setuphead[section][
  align=middle,
  number=no,
  style={\rm\setcharacterkerning[capitals]\WORD},
  before={\blank[line,force]},
  after={\blank[line]}
]

\setuphead[subsection][
  align=middle,
  number=no,
  style={\tf\sc\word},
  before={\blank[line,force]},
  after={\blank[line]}
]

\setuplist[chapter][
  alternative=c,
  interaction=text,
  style={\word\sc},
]
\setuplist[section,subsection][
  alternative=a,
  style=\tfx\italic,
  interaction=text,
  margin=2em,
  numberstyle=,
  textstyle=,
  numberstyle=\tfx,
]

\setuplist[subsection][
  margin=4em,
]

\setuplistalternative

\definecharacterkerning [capitals] [factor=.05]

\definefontfeature [default][default][
  protrusion=quality,
  expansion=quality,
  %mode=node,
  script=latn,
  onum=yes,
  %dlig=yes,
  liga=yes,
]

\definefontfeature [smallcaps] [default] [smcp=yes]
\def\sc{\addff{smallcaps}\setcharacterkerning[capitals]}

\setupbodyfontenvironment [default] [em=italic]

\starttypescript [serif] [bukyvede]
  \setups [font:fallback:serif]
  \definefontsynonym [Serif]        [name:Bukyvede]         [features=default]
  \definefontsynonym [SerifItalic]  [name:Bukyvede-Italic]  [features=default]
\stoptypescript
\usetypescript [bukyvede]
\definetypeface [hlaholice] [rm] [serif] [bukyvede]     [default] [encoding=ec]
\definetypeface [cyrilice]  [rm] [serif] [bukyvede]     [default] [encoding=ec]
\definetypeface [lmstd]     [rm] [serif] [latin-modern] [default] [encoding=texnansi]

\usetypescriptfile[type-cmu]
\usetypescript[computer-modern-unicode]
\setupbodyfont[computer-modern-unicode,9pt]

\usetypescript  [serif]   [hz] [highquality]
\setupalign     [hanging,hz]

\usemodule[bib]
\usemodule[transliterator]

\setupcite[authoryear][compress=no]

\setuppublications[%
  alternative=apa,%
  refcommand=authoryear,%
  sorttype=bbl,%
  numbering=yes,%
  autohang=yes%
]%

\setuppublicationlist[%
  artauthor=\invertedauthor%
]

% == REFERENCES ===============================================================

\startpublication[
  k=aks,
  t=book,
  a={{Birnbaum/Schaeken}},
  y=1999,
  n=4,
  u=http://www.schaeken.nl/lu/research/online/publications/akslstud/index.htm,
  s={Studien},
]
\author[]{Henrik}[H.]{}{Birnbaum}
\author[]{Jos}[J.]{}{Schaeken}
\pubyear{1999}
\title{Altkirchenslavische Studien}
\volume{2}
\city{München}
\stoppublication

\startpublication[
  k=bornemann,
  t=book,
  a={{Bornemann/Risch}},
  y=1978,
  n=2,
  s={Grammatik},
]
\author[]{Eduard}[]{}{Bornemann}
\author[]{Ernst}[]{}{Risch}
\pubyear{1978}
\title{Griechische Grammatik}
\city{Frankfurt am Main}
\edition{2.}
\stoppublication

\startpublication[
  k=bh,
  t=book,
  a={{Bringhurst}},
  y=2008,
  n=4,
  s={Bringhurst},
]
\author[]{Robert}[R]{}{Bringhurst}
\pubyear{2008}
\title{The Elements of Typographic Style}
\edition{3.2}
\city{Point Roberts WA, Vancouver}
\stoppublication

\startpublication[
  k=dintb,
  t=book,
  a={{DIN}},
  y=2001,
  n=5,
  s={DIN},
]
\editor[]{}[]{}{DIN Deutsches Institut für Normung e.~V.}
\pubyear{2001}
\title{Bibliotheks und Dokumentationswesen}
\city{Berlin/Wien/Zürich}
\stoppublication

\startpublication[
  k=duden,
  t=book,
  a={{Drosdowski/Müller/Scholze-Stubenrecht/Wermke}},
  y=1952,
  n=1,
  s={DUDEN},
]
\editor[]{Günther}[]{}{Drosdowski}
\editor[]{Wolfgang}[]{}{Müller}
\editor[]{Werner}[]{}{Schulze-Stubenrecht}
\editor[]{Matthias}[]{}{Wermke}
\pubyear{1991}
\title{DUDEN Rechtschreibung der deutschen Sprache}
\city{Mannheim et al}
\edition{20.}
\stoppublication

\startpublication[
  k=kirschbaum,
  t=book,
  a={{Kirschbaum}},
  y=2001,
  n=3,
  s={Grammatik},
]
\author[]{Ernst Georg}[]{}{Kirschbaum}
\pubyear{2001}
\title{Grammatik der russischen Sprache}
\city{Berlin}
\stoppublication

\startpublication[
  k=iso,
  t=inbook,
  a={{ISO}},
  y=1995,
  n=6,
  s={ISO~9},
]
\editor[]{}[]{}{{{\sc iso} International Organization for Standardization}}
\pubyear{1995}
\title{Information and documentation -- Transliteration of Cyrillic characters into Latin characters -- Slavic and non-Slavic languages}
\edition{2.}
\crossref{dintb}
\pages{230--245}
\stoppublication

%==============================================================================

\setupframed[
  frame=off,
  align=normal,
  location=top,
]

\defineframed[displayouter][
  location=top,
  align={normal,verytolerant},
  frame=off,
  style=\tfx,
]
\defineframed[displayinner][displayouter][
  offset=1ex,
  width=.47\textwidth,
]

\definenumber[excnt]
\setnumber[excnt][1]

% This should rather be done using key-value args but I'm too lazy now.
% 1: mode; 2: hyphenate original; 3: hyphenate transliteration;
% 4: font for original; 5: caption; 6: original text.
\def\trlex#1#2#3#4#5#6{%
  \setuplocalinterlinespace[line=8pt]%
  \startplacefigure [
    location=force,
    title={\type{[mode=#1,hyphenate=#3]}\hskip 1em{\italic #5}}
  ]%
    \displayouter{%
      \displayinner{%
        \setupbodyfont[#4]%
        \tfx
        %\setuptolerance[verytolerant, stretch]
        \setuptolerance[verytolerant]
        \unskip\language[#2]#6\par
      }%
      \displayinner{%
        \tfx
        \transliterate[mode=#1,hyphenate=#3]{#6\par}%
      }
    }
  \stopplacefigure
  \incrementnumber[excnt]%
}

\defineframedtext[CenteredText][width=fit,frame=off,align=middle]

\usemodule[int-load]
\loadsetups[t-transliterator.xml] 


\setupwhitespace[medium]
\language[en]

\starttext

\setuppagenumbering[state=stop]

\blank[3cm,force]


%\showframe
\startstandardmakeup[location=middle]

\setuplayout[width=middle]
\raggedcenter
\vfill
  {\setupbodyfont[19pt]
  {\em The}
  \blank [2*big]
  {\tfc\sc transliterator}
  \blank [2*big]
  {\em for \CONTEXT}
  \blank [9*big]
  {\tfa\sc manual}
  }
\vfill
\stopstandardmakeup

\startstandardmakeup
\vfill
\framed [frame=off,topframe=on] {%
\tfxx\ss\setupinterlinespace[small]%
\startlines
The {\em Transliterator} module and mini-manual,
by Philipp Gesang, Radebeul.
Mail any patches or suggestions to

{\tt philipp -dot- gesang -at- alumni -dot- uni-heidelberg -dot- de}
\useurl[me][https://phi-gamma.net]
\from[me]%
\stoplines
}
\stopstandardmakeup

\setuppagenumbering[%
  location=middle,
  state=start,
  style=\tfc
]

\setuppagenumber[number=1]
\completecontent
\chapter{Usage and Functionality}
\section{Overview}
The Transliterator provides two commands: \type{\setuptransliterator}
preferably goes into the preamble and allows for global configuration.
The Transliterator is invoked locally by \type{\transliterate} which does the
actual transliteration of text passages.

\setup{setuptransliterator}

\setup{transliterate}

\section{Loading and Configuring the Module}
In order to use the Transliterator in a document we put the following somewhere before
\type{\starttext}.
\starttyping
\usemodule[transliterator]
\stoptyping
Although it has some defaults already set at this point they will most likely
not correspond to what is needed in the document.
To override the presets we use the command \type{\setuptransliterator[#1]}.
It takes a comma separated list of two key-value pairs: \type{mode} and
\type{hyphenate}.
Through {\em mode} we specify the transliteration method.
By the time of this writing this can be one of the following set:

\startplacetable[location=top,title=Transliteration modes.]
  \tfx
  \starttabulate[|l|p|]
    \HL
    \NC mode \NC description \NC\NR
    \HL
    \NC \type{all}              \NC {\sc iso}~9 complete \NC\NR
    \NC \type{bg_de}            \NC Bulgarian, German „scientific“ transliteration\NC\NR
    \NC \type{gr}               \NC transliteration for Greek \NC\NR
    \NC \type{gr_n}             \NC transliteration for Greek obeying nasalizations \NC\NR
    \NC \type{iso9_ocs}         \NC == \type{all} plus non-{\sc iso} additions for Old (Church) Slavonic \NC\NR
    \NC \type{ocs}              \NC “scientific” transliteration for Old (Church) Slavonic\NC\NR
    \NC \type{ocs_cz}           \NC Czech transcription for Old (Church) Slavonic\NC\NR
    \NC \type{ocs_gla}          \NC “scientific” transliteration for Old (Church) Slavonic / Glagolitic alphabet\NC\NR
    \NC \type{ru}               \NC {\sc iso}~9 Russian \NC\NR
    \NC \type{ru_cz}            \NC Czech transcription for Russian\NC\NR
    \NC \type{ru_old}           \NC {\sc iso}~9 Russian plus pre-1918 chars (the default)\NC\NR
    \NC \type{ru_transcript_de} \NC German transcription for Russian \NC\NR
    \NC \type{ru_transcript_en} \NC English transcription for Russian \NC\NR
    \NC \type{sr_tocy}          \NC Serbian, Latin to Cyrillic \NC\NR
    \NC \type{sr_tolt}          \NC Serbian, Cyrillic to Latin \NC\NR
    \HL
  \stoptabulate
\stopplacetable


{\em Nota bene}: The description at this point only serves as a placeholder as the
transliteration modes are discussed in detail later in this document.

Through the \type{hyphenate} argument it is possible to adjust the language
that is used for hyphenation.
Specifying \type{\setuptransliterator[hyphenate=nl]} will let every transliterated
part of the document be processed according to dutch rules, leaving the overall
\type{\language[#1]} configuration unchanged for the rest of the content.

Another argument, \type{deficient_font} can be used in
combination with the modes \type{all}, \type{ru_old} and
\type{iso9_ocs}. It lets you circumvent the deficiency that some
fonts show concerning the characters that {\sc iso}~9 assigns to
cyrillic “ь” and “ъ”. Set it to {\em true} to enable it.

The actual transliteration is done using the macro
\type{\transliterate[#1]} \type{{#2}}.
The second argument takes the raw string in the original language that we want
to process, while the first, optional argument accepts local adjustments for
\type{mode} and \type{hyphenate}.
Thus, we would typeset one of Epicuros' sayings like this:
{\setuptolerance[verytolerant]
\starttyping
\transliterate[mode=gr]{κακὸν ἀνάγκη, ἀλλ' οὐδεμία ἀνάγκη ζῆν 
  μετὰ ἀνάγκης}
\stoptyping
\noindentation which yields \quotation{\transliterate[mode=gr]{κακὸν ἀνάγκη, ἀλλ' οὐδεμία ἀνάγκη ζῆν
μετὰ ἀνάγκης}} in the {\sc pdf} output.
}
Alternatively there is an environment, \type{\starttransliterate[#1]}, as well,
that takes the same arguments.

There are two special switches for the {\em Serbian} patterns,
\type{hinting} and \type{sr_exceptions}, allowing for a little
more fine-tuning.
If activated, hinting provides the special character “\type{*}” as
a means to indicate positions, where the sequences “lj” and “nj”
are to be treated as separate consonants.
E.~g. \type{\transliterate[mode=sr_tocy]{in*jekcija}} is
correctly transliterated as \transliterate[mode=sr_tocy]{in*jekcija},
and not \transliterate[mode=sr_tocy,sr_exceptions=no]{injekcija}.
Likewise, further exceptions that are internally represented as
a lookup table can be toggled off or on by the
\type{sr_exceptions} switch.
This pertains to words like “nadživeti” (result: \transliterate[mode=sr_tocy]{nadživeti})
but may lead to accidental false positives in cases that the
module author didn’t foresee.
By default both hinting and lexical exceptions are set to
\type{yes}.

For orientation purposes the Transliterator comes with two macros that allow
for closer inspection of the internal tables.
\type{\showOneTranslitTab{#1}} outputs, obviously, a single table; their
identifiers
can be found in the \type{trans_}
\type{tables_*.lua} files in the transliterator
directory.
The lazy alternative is \type{\showTranslitTabs} which prints all registered
tables in a row nicely formatted as indexable sections.
(Be warned, this may take some time.)

\chapter{Introduction}

\hfil\framed[width=\hsize,align=left]{%
  \inframed[bottomframe=on]{\it What's all this, then?}
  \blank[medium]
  {\sc Graham Chapman}
}
\blank[2*big]

\noindentation  At the first glance, {\em transliteration} -- the accurate representation of letters from one
alphabet in another -- seems obsolete after the advent of Unicode
which made its way even into \TeX\ lately.
Why not just go on and write down everything in the original script?
But still there are lots of situations where transliteration is desirable,
e.~g. some scholarly habits might prescribe it in the main text with citations in
footnotes left in the original alphabet; or transliteration might alleviate
comparison within one language that happens to be written in different scripts;
finally, including text in a foreign script might be impossible if there is no
appropriate font which fits the main text.
However, it is still most convenient for the writer to keep the
untransliterated original in the document source as this allows for reusing it in
another context where different transliterations rules might apply.
The Transliterator module is meant to provide both: have the original in the
source and a transliteration only in the final document.

Another way of handling foreign languages is {\em transcription}.
It aims at producing some representation that does not rely on symbolisms
alien to the language and thus to be at least \quotation{pronouncable}
without further know\-ledge.
As transcription methods are language specific and highly idiosyncratic they
complicate the restoration of the original phrase because information may be lost.
The Transliterator provides means of transcription as well but in most cases
you should refrain from using them (\type{[mode=ru_transcript_en]},
\type{[mode=ru_transcript_de]}). 

For Cyrillic scripts the best quality is achieved using the standardized
transliteration according to {\sc iso~9}.\footnote{\cite[authoryear][iso].}
This method not only covers all contemporary languages that are written in
a variety of Cyrillic but provides a bijective mapping on latin characters as
well.
Consequently, you can unambiguously revert the transliteration into
its original form which was impossible with previous versions of {\sc
iso}~9 because
they contained several exceptions depending on the original language.
Although fifteen years old it has not yet made its way into scholarly
publications at large so it might not immediately look familiar.\footnote{
  A hasty glance at the latest issues of around 20~journals in a local library
  revealed that 2~of them actually are using {\sc iso}~9, these are {\em Przegląd
  wschodni} as of Nr. X, 3 (2008) and {\em Kwartalnik historyczny} as of CXVI,
  3 (2009); the latter even contains a table on p.~218 showing a subset of the
  {\sc iso}~9 transliteration rules.
}
The diacritics are not identical to the \quotation{scientific}
transliteration used in Slavic studies but as long as your editor does not
enforce its traditional method you should always prefer {\sc iso}~9
(\type{[mode=ru]}, \type{[mode=ru_old]}, \type{[mode=all]}).

But {\sc iso}~9, too, has its shortcomings.
It has no definitions for historical forms of the cyrillic script like 
pre-XVIII-century Russian and Old (Church) Slavonic while those are covered by
the scholarly transliterations.
To amend the situation the Transliterator provides an extension to {\sc
iso}~9 for
Old Slavonic containing the glyphs 
\startluacode
local translit = thirddata.translit
environment.loadluafile("trans_tables_scntfc")
local cnt, len = 0, 0 
for i,j in pairs(translit.ocs_add_low) do
  len = len + 1
end

for k,v in pairs(translit.ocs_add_low) do
  cnt = cnt + 1
  context.bgroup() 
    context.setupbodyfont({"cyrilice"})
    context(k)
  context.egroup() 
  if cnt < len -1 then
    context(", ") 
  elseif cnt < len then
    context("\\ and ")
  end
end
\stopluacode
\ taken from the scientific transliteration (\type{[mode=iso9_ocs]}).
If you prefer more coherency you might want to use pure \quotation{scientific}
transliteration (\type{[mode=ocs]}).
This method is complemented by \type{[mode=ocs_gla]}, the only option the
Transliterator offers for the Glagolitic alphabet; they can be used consistently
along each other as they were taken from the same
book.\footnote{\cite[authoryear][aks] p.~77 \cite[url][aks].}

As far as I know there is no standardized transliteration for Greek so I had to
resort to the one that is used in scholarly literature.
Its main drawback is that it has no representation for diacritics apart from
(rough) breathing, but it respects specific rules for diphthongs and vowels in
initial positions (\type{[mode=gr]}).
There is one alternative mode for those who prefer their {\em γ} phonetically
resolved to /{\em n}/ before velars ({\em γ}, {\em κ}, {\em χ} and {\em ξ};
\type{[mode=gr_n]}).

Concerning the hyphenation within transliterated passages the default is set to
to \type{[hyphenate=cs]} (Czech) which produces reasonable results when using
\type{all}, \type{iso9_ocs} or \type{ru_cz}.
For stuff like the English and German transcription use their respective native
hyphenation.\footnote{%
  You'll have to specify this through \type{\setuptransliterator}
  or locally because the default hyphenation is {\em not} the same as your
  documents'.
}
However, as there is no hyphenation pattern I know of that closely resembles the
transliteration of Greek you might have to resort to putting \type{\discretionary}
hyphens when line breaking does not satisfy.

The Transliterator as a whole is nothing more than a bunch of dictionaries
containing substitution rules for tokens that may occur in the text.
These tokens may be single characters or strings of more than one character.
As there is no simple way to impose order onto those dictionaries the rules for
one transliteration method are, if needed, distributed over more than one table
which will be applied successively to ensure that multi-character rules
are processed first.


\setupfloats[spacebefore=small,spaceafter=small]
\startplacetable[location=left,title={
  Processing time for corpus {\language[cs]Evgenij Onegin} according to
  GNU time(1) and the \CONTEXT\ stats.
}]
  \starttabulate[|l|cg(.)|cg(.)|]
    \HL%····················································%
    \NC mode \NC time(1) in $s$ \NC \CONTEXT \NC \NR
    \NC <none>                  \NC  8.98 \NC  8.82 \NC \NR
    \NC \type{all}              \NC  8.37 \NC  8.25 \NC \NR
    \NC \type{ru_cz}            \NC  8.61 \NC  8.48 \NC \NR
    \NC \type{ru_transcript_en} \NC  9.26 \NC  9.10 \NC \NR
    \NC \type{ru_transcript_de} \NC 14.83 \NC 14.71 \NC \NR
    \HL%····················································%
  \stoptabulate
\stopplacetable
\setuptolerance[tolerant]
Following suggestions from the mailing list, the Transliterator uses {\em LPeg}
when substituting.
This means a huge speed improvement for most substitution modes when compared
to the older mechanism that used \type{string.gsub} iteratively.
In ordinary use when transliterating single words or short phrases the
Transliterator should have little impact on document processing time at large,
with the exception of the German transcription mode, perhaps.\footnote{
  The problem lies within the rule set for the German transcription which
  dictates different instructions depending on the environment of a character;
  these may conflict, i.~e. it is impossible to substitute a character stream
  in a single run as some rules may apply only to the result of previous rule.
  Let me know if there's a way to tell LPeg to backtrack to the last character
  of a match and not to continue on the next.
}
Transliterating (and typesetting in MKIV) \transliterate{Александр Пушкин}'s verse novel
\transliterate{Евгений Онегин}, a corpus of about 27000 words, in
\type{[mode=all]} shows little to no delay at all.
In fact, typesetting cyrillic letters with russian hyphenation seems slow
things down so much that transliteration may be faster and uses slightly less
memory.\footnote{%
  On an IBM T43: \tt 2.6.32-ARCH \#1 SMP PREEMPT Tue Feb 9 14:46:08 UTC 2010
  i686 Intel(R) Pentium(R) M processor 1.60GHz GenuineIntel GNU/Linux.
}




\chapter[ex]{Examples}
\section{Cyrillic scripts}
\subsection{{\sc iso}~9 and derivatives}

Several transliteration rules are either strictly {\sc iso}~9 compliant
(\type{ru}, \type{ru_old}, \type{all}) or contain {\sc iso}~9 as a
subset (\type{iso9_ocs}).\footnote{%
  Unfortunately \CONTEXT\ still lacks language files for some of them
  so please excuse the inadequate hyphenation in these cases.%
}

\trlex{ru}{ru}{cs}{computer-modern-unicode}{%
  Transliteration rules for the contemporary russian alphabet.%
}{%
  В~ворота гостиницы губернского города NN въехала довольно красивая рессорная
  небольшая бричка, в~какой ездят холостяки: отставные подполковники,
  штабс-капитаны, помещики, имеющие около сотни душ крестьян, — словом, все те, 
  которых называют господами средней руки. 
  В~бричке сидел господин, не красавец, но и~не дурной наружности, ни слишком
  толст, ни слишком тонок; нельзя сказать, чтобы стар, однако ж~и~не так чтобы
  слишком молод.
}

\trlex{ru_old}{ru}{cs}{computer-modern-unicode}{%
  With aditional characters for pre-1981 Russian orthography (100~per
  cent {\sc iso}~9).%
}{%
  А~сведется віра, убьютъ сотцкого в~селѣ, ино тебѣ взяти полтіна, а~не
  сотцкого,
  ино четырѣ гривны, а~намъ віръ не таити в~Новѣгородѣ; а~о~убіствѣ віръ нѣтъ.
  А~что волости, честны король, новгородцкіе, ино тебѣ не держати своими мужи,
  а~держати мужми новогородцкими.
  А~что пошлина в~Торжку и~на Волоцѣ, тівунъ свои держати на своеи чясті,
  а~Новугороду на своеи чясти посадника держаті.
  А~се волости новогородцкіе: Волокъ со всѣми волостми, Торжокъ, Бѣжіці,
  Городець
  Палець, Шіпинъ, Мелеця, Егна, Заволочье, Тиръ, Пермь, Печера, Югра, Вологда
  с~волостмі.
}

\trlex{all}{ru}{cs}{computer-modern-unicode}{%
  The complete cyrillic mapping from {\sc iso}~9; transliterating Belarusian.%
}{%
  Беларуская мова, мова беларусаў, уваходзіць у~сям’ю індаеўрапейскіх моў, яе
  славянскай групы і~ўсходнеславянскіх моваў падгрупы, на якой размаўляюць
  у~Беларусі і~па ўсім свеце, галоўным чынам у~Расіі, Украіне, Польшчы.
  Б.~м. падзяляе шмат граматычных і~лексічных уласцівасцяў з~іншымі
  ўсходнеславянскімі мовамі (гл. таксама: Іншыя назвы беларускай мовы і~Узаемныя
  ўплывы усходнеславянскіх моваў).
}

\trlex{all}{uk}{cs}{computer-modern-unicode}{%
  The complete cyrillic mapping from {\sc iso}~9; transliterating Ukrainian.%
}{%
  Украї́нська мова (застарілі назви -- руська мова, проста мова […]) --
  слов'янська мова, державна в~Україні та одна з~трьох «офіційних мов на рівних
  засадах» у~не\-ви\-зна\-ній Придністровській Молдавській Республіці.
  За різними оцінками загалом у~світі українською мовою говорить від 41~млн.
  до 45~млн. осіб, вона входить до третього десятка найпоширеніших мов
  світу.
}

\trlex{all}{ru}{cs}{computer-modern-unicode}{%
  The complete cyrillic mapping from {\sc iso}~9; transliterating Serbian.%
}{%
  Српски језик је један од словенских језика из породице индоевропских језика.
  Први писани споменици у~српској редакцији старословенског језика потичу из XI
  и~XII века.
  Српски језик је стандардни језик у~службеној употреби у~Србији, Босни
  и~Херцеговини и~Црној Гори, а~у~употреби је и~у другим земљама гдје живе
  Срби, међу осталима и~у~Хрватској.
}

\trlex{iso9_ocs}{ru}{cs}{cyrilice}{%
  Transliteration rules according to {\sc iso}~9 with additions for Old (Church)
  Slavonic.%
}{%
  Что сѧ дѣѥтѣ по вѣремьнемь~: то ѿидето по вѣрьмьнемь~: приказано бѹдѣте
  добрымъ людѣмъ~: а любо грамотою ѹтвѣрдѧть~: како то бѹдѣте всемъ вѣдомъ~:
  или кто посль живыи ѡстанѣть сѧ~: того лѣт͠ коли алъбрахтъ~: влд͠ка ризкии
  ѹмьрлъ~: ѹздѹмалъ кнѧзѣ смольнескыи~: мьстиславъ~: двд͠въ сн͠ъ~: прислалъ въ
  ригѹ своѥго лѹчьшего попа~: ѥрьмея~: и съ нимь ѹмьна мѹжа пантелья~:
  исвоѥго горда смольнеска~: та два была послъмь ѹ ризѣ~: из ригы ѥхали на
  гочкыи берьго~: тамо твердити миръ~:
}

\subsection{“Scientific” transliteration}
These transliterations are widely used among scholars, mainly linguists and, to
a lesser extent, historians.
They comprise large character sets in order to represent the original text
adequately and facilitate comparison of texts of the same language written in
different scripts; they are not, however, as easily reversible as {\sc
iso}~9.

\trlex{ocs}{ru}{cs}{cyrilice}{%
  Transliteration for Old Slavonic used in Slavic studies, taken from the
  excellent book of \cite [authoryear][aks].\footnote{%
    This one and both of the following Czech transliterations, although
    elegantly dealing with hard and weak signs by taking characters from the
    Cyrillic alphabet, are not unquestioned from a typographical point of
    view:
    \quotation{If contrasting faces are used for phonetic transcriptions and
    main text, each entire phonetic word or passage, not just the individual
    phonetic characters, should be set in the chosen phonetic face.  Patchwork
    typography, in which the letters of a single word come from different faces
    and fonts, is a sign of typographic failure. […]
    Such mixtures are almost sure to fail unless all the fonts involved have
    been designed as a single family.}
    (\cite [authoryear][bh])
    From this follows that it is advisably to reconsider your font whether it indeed
    provides the needed glyphs from Russian as well.
  }%
}{%
  Се начнемъ повѣсть сию. 
  По потопѣ . первиє снве Ноєви . раздѣлиша землю . Симъ . Хамъ . Афетъ . и~ꙗсѧ
  въстокъ . Симови Персида . Ватрь . тоже  и~до Индикиꙗ в~долготу и~в~ширину [и
  до Нирокоуриа] ꙗкоже рещи ѿ въстока и~до полуденьꙗ . и~Суриꙗ .
  и~Индиа по Єфратъ рѣку . Вавилонъ . Кордуна . Асурѧне . Мисопотамира .
  Аравиꙗ . старѣишаꙗ . Єлмаисъ . Инди . Равиꙗ . на всѧ  Д.
}

\trlex{ru_cz}{ru}{cs}{computer-modern-unicode}{%
  Czech phonetic transcription for contemporary Russian.%
}{%
  Прошло семь лет после 12-го года. Взволнованное историческое море Европы
  улеглось в свои берега. Оно казалось затихшим; но таинственные силы,
  двигающие человечество (таинственные потому, что законы, определяющие их
  движение, неизвестны нам), продолжали свое действие.
  Несмотря на то, что поверхность исторического моря казалась неподвижною, так
  же непрерывно, как движение времени, двигалось человечество. Слагались,
  разлагались различные группы людских сцеплений; подготовлялись причины
  образования и~разложения государств, перемещений народов.%
}

\trlex{ocs_cz}{ru}{cs}{cyrilice}{%
  Czech phonetic transcription for Old Slavonic (superset of the corresponding
  Russian transcription).
}{%
  Убьеть мужь мужа, то мьстить брату брата, или сынови отца, любо отцю сына,
  или братучаду, любо сестрину сынови; аще не будеть кто мьстіѧ, то 40 гривенъ
  ꙁа голову; аще будеть русинъ, любо гридинъ, любо купчина, любо іѧбетник, любо
  мечникъ, аще иꙁъгои будеть, любо словенинъ, то 40 гривенъ положити ꙁа нь.
}

\subsection{Serbian}
The tables for converting Serbian text between Cyrillic and Latin
alphabets are \type{sr_tolt} and \type{sr_tocy}.
\trlex{sr_tolt}{sr}{hr}{computer-modern-unicode}{%
  Transliteration ћирилица \rightarrow\ латиница.%
}{%
  Српски језик је један од словенских језика из породице
  индоевропских језика. Први писани споменици у српској редакцији
  старословенског језика потичу из XI и XII века.

  Српски језик је стандардни језик у службеној употреби у Србији,
  Босни и Херцеговини и Црној Гори, а у употреби је и у другим
  земљама где живе Срби, међу осталима и у Хрватској.%
}

\trlex{sr_tocy}{hr}{sr}{computer-modern-unicode}{%
  Transliteration latinica \rightarrow\ ćirilica.%
}{%
  Srpski jezik je jedan od slovenskih jezika iz porodice
  indoevropskih jezika. Prvi pisani spomenici u srpskoj
  redakciji staroslovenskog jezika potiču iz XI i XII veka.

  Srpski jezik je standardni jezik u službenoj upotrebi u Srbiji,
  Bosni i Hercegovini i Crnoj Gori, a u upotrebi je i u drugim
  zemljama gde žive Srbi, među ostalima i u Hrvatskoj.%
}

\subsection{Bulgarian}

\trlex{bg_de}{bg}{cs}{computer-modern-unicode}{%
  German scientific transliteration for Bulgarian (based on old {\sc
  iso}~9 standard).%
}{%
  Българският език е индоевропейски език от групата на
  южнославянските езици. Той е официалният език на Република
  България и един от 23-те официални езика на Европейския съюз.
}

\subsection{Legacy national transcriptions}
At the moment there are tables for “old school” transcription into three
languages: English (via \type{ru_transcript_en}), German
(\type{ru_transcript_de}) and Czech (\type{ocs_cz}).
At least the German one is almost unreadable if used with
strings longer than two words.
As we have the bijective {\sc iso}~9 mapping at hand there should be no reason at all
to use any of them.

\trlex{ru_transcript_en}{ru}{en}{computer-modern-unicode}{%
  English transcription for contemporary Russian.%
}{%
  Прошло семь лет после 12-го года. Взволнованное историческое море Европы
  улеглось в свои берега. Оно казалось затихшим; но таинственные силы,
  двигающие человечество (таинственные потому, что законы, определяющие их
  движение, неизвестны нам), продолжали свое действие.
  Несмотря на то, что поверхность исторического моря казалась неподвижною, так
  же непрерывно, как движение времени, двигалось человечество. Слагались,
  разлагались различные группы людских сцеплений; подготовлялись причины
  образования и~разложения государств, перемещений народов.%
}

\trlex{ru_transcript_de}{ru}{deo}{computer-modern-unicode}{%
  German transcription for contemporary Russian.\footnote{%
    Following \cite[authoryear][duden] p.~82; all the canonical rules are
    implemented save one: {\em -его} and {\em -ого} should resolve to {\em
    -ewo} and {\em -owo} respectively iff genitive endings.
    As this is a grammatical rather than graphetical criterion writing  a
    substitution algorithm would amount to do natural language parsing.
    To make things worse this rule is phonetically confused as it would not
    take care of other contexts where {\em г} in those patterns is articulated
    as /{\em v}/ like for instance in {\em сегодня} (which is a historical
    genitive, though …).
    So even if this could be implemented it would not be advisable to use such
    a rule.%
  }%
}{%
  Прошло семь лет после 12-го года. Взволнованное историческое море Европы
  улеглось в свои берега. Оно казалось затихшим; но таинственные силы,
  двигающие человечество (таинственные потому, что законы, определяющие их
  движение, неизвестны нам), продолжали свое действие.
  Несмотря на то, что поверхность исторического моря казалась неподвижною, так
  же непрерывно, как движение времени, двигалось человечество. Слагались,
  разлагались различные группы людских сцеплений; подготовлялись причины
  образования и~разложения государств, перемещений народов.%
}

\section{Glagolitic}
\trlex{ocs_gla}{ru}{cs}{hlaholice}{%
  “Scientific” transliteration for Old Slavonic written in the Glagolitic
  alphabet as used in \cite[authoryear][aks].%
}{%
  [ⰲⰾ] 
  ⰰⰴⱏⰻⰽⱁ ⱍⰽ҃ⱏ ⱄⰻ ⱈⱁⱋⰵⱅⱏ ⱃⰰⰸ[ⱁⱃⰻⱅ] 
  ⰻ ⰸⰰⰽⱁⱀⱏ ⰿⰰⱀⰰⱄⱅⱏⰻⱃⱏⱄⰽⰻ: [ⰻⰶⰵ] 
  ⱅⱏⰻ ⱆⱄⱅⰰⰲⰻ჻ Ⱃⰵⱍⰵ ⰶⰵ ⰻⰳⱆⰿ[ⱏ] [ⱀⱏ] 
  ⰽⰰⰽⱁ ⱈⱁⱋⰵⱅⱏ ⱃⰰⰸⱁⱃⰻⱅⰻ ⰸⰰⰽ[ⱁⱀⱏ] 
  [.] [ⰰ] ⰵⱄⱅⱏ· ⱍⱃⱏⰲⰻ⁖ ⰻ [ⰿ] [..........] 
  [..] ⰿⱏ ⱀⰵ ⰿⱁⰶⰵⰿⱏ ⱄⰵⰳⱁ ⱅⱃⱏⱂⱑⱅ[ⰻ] 
  [ⰴⰰ] ⰾⱆⰱⱁ ⱄⰵⰳⱁ ⰻⰿⱑⰻ ⱄⱏⰴⱑ჻ ⰰ ⰿⱏⰻ ⱁ 
  [ⱅⰻ]ⰴⰵⰿⱏ: ⰾⱆⰱⱁ ⱄⰵⰳⱁ ⱂⱆⱄⱅⰻ: ⰴⰰ ⱁⱅ 
  [ⰻⰴ]ⰵⱅⱏ ⰻⰶⰵ ⰵⱄⱅⱏ ⱂⱃⰻⱎⱏⰾⱏ: ⱄ[ⰵ] 
}

\section{Greek}
The Transliterator offers two modes for handling Greek: \type{gr} and
\type{gr_n}.
They differ only on one aspect.
\type{gr} transliterates the canonical Greek alphabet as well as the
special glyphs Digamma, Quoppa and Sampi.
\type{gr_n} behaves exactly the same way except that nasalization is observed
such that \type{γ+[γ|κ]} yields \type{n+[g|k]}.

\trlex{gr}{agr}{de}{computer-modern-unicode}{%
  Transliteration for Greek -- standard.
}{%
  οἴνῳ δὲ κάρτα προσκέαται, καί σφι οὐκ ἐμέσαι ἔξεστι, οὐκὶ οὐρῆσαι ἀντίον
  ἄλλου.
  ταῦτα μέν νυν οὕτω φυλάσσεται, μεθυσκόμενοι δὲ ἐώθασι βουλεύεσθαι τὰ
  σπουδαιέστατα τῶν πρηγμάτων: τὸ δ᾽ ἂν ἅδῃ σφι βουλευομένοισι, τοῦτο τῇ
  ὑστεραίῃ νήφουσι προτιθεῖ ὁ στέγαρχος, ἐν τοῦ ἂν ἐόντες βουλεύωνται, καὶ ἢν
  μὲν
  ἅδῃ καὶ νήφουσι, χρέωνται αὐτῷ, ἢν δὲμὴ ἅδῃ, μετιεῖσι. τὰ δ᾽ ἂν νήφοντες
  προβουλεύσωνται, μεθυσκόμενοι ἐπιδιαγινώσκουσι.
}%

\trlex{gr_n}{agr}{de}{computer-modern-unicode}{%
  Transliteration for Greek -- alternative respecting nasalization.
}{%
  ταῦτα καὶ νεωτέρῳ καὶ πρεσβυτέρῳ ὅτῳ ἂν ἐντυγχάνω ποιήσω, καὶ ξένῳ καὶ ἀστῷ,
  μᾶλλον δὲ τοῖς ἀστοῖς, ὅσῳ μου ἐγγυτέρω ἐστὲ γένει.
}%
   

\chapter{References}
%\cite[authoryear][iso]
\nocite[duden]
\nocite[bornemann]
\nocite[kirschbaum]
\nocite[iso]
\nocite[aks]
\nocite[dintb]
\placepublications [criterium=all]

\stoptext
%   vim:ft=context
