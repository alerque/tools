% language=uk

\environment workflows-style

\startcomponent workflows-xml

\startluacode
  for i=1,10 do
    local filename = string.formatters["temp-%02i.xml"](i)
    local filedata = string.formatters["<?xml version='1.0'?><p>snippet %i</p>"](i)
    io.savedata(filename,filedata)
  end
\stopluacode

\startchapter[title={XML}]

When you have an \XML\ project with many files involved, finding the right spot
of something that went wrong can be a pain. In one of our project the production of
some 50 books involves 60.000 \XML\ files and 20.000 images. Say that we have the
following file:

\startbuffer[demo]
<?xml version='1.0'?>
<document>
  <include name="temp-01.xml"/> <include name="temp-02.xml"/>
  <include name="temp-03.xml"/> <include name="temp-04.xml"/>
  <include name="temp-05.xml"/> <include name="temp-06.xml"/>
  <include name="temp-07.xml"/> <include name="temp-08.xml"/>
  <include name="temp-09.xml"/> <include name="temp-10.xml"/>
</document>
\stopbuffer

\typebuffer[demo]

Before we process this file we will merge the content of the files defined
as includes into it. When this happens the filename is automatically
registered so it can be accessed later.

\startbuffer
\startxmlsetups xml:initialize
    \xmlincludeoptions{#1}{include}{filename|name}{recurse,basename}
    \xmlsetsetup{#1}{p|document}{xml:*}
\stopxmlsetups

\startxmlsetups xml:document
    \xmlflush{#1}
\stopxmlsetups

\startxmlsetups xml:p
    \inleftmargin{\infofont\xmlinclusion{#1}}
    \xmlflush{#1}
    \par
\stopxmlsetups

\xmlregistersetup{xml:initialize}

\xmlprocessbuffer{main}{demo}{}
\stopbuffer

\typebuffer

In this case we put the name of the file in the margin. Depending on when and how
elements are flushed other solutions, like overlays, can be used.

\startpacked
\getbuffer
\stoppacked

At any moment you can see what the current node contains. The whole (merged)
document is also available:

\startbuffer
\xmlshow{main}
\stopbuffer

\typebuffer

A small font is used to typeset the (sub)tree:

\blank \getbuffer \blank

You can also save the tree:

\startbuffer
\xmlsave{main}{temp.xml}
\stopbuffer

\typebuffer \getbuffer

This file looks like:

\typefile{temp.xml}

\stopchapter

\stopcomponent
